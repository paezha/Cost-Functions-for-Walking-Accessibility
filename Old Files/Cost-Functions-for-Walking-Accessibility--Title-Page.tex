\documentclass[]{elsarticle} %review=doublespace preprint=single 5p=2 column
%%% Begin My package additions %%%%%%%%%%%%%%%%%%%
\usepackage[hyphens]{url}

  \journal{Journal of Transport Geography} % Sets Journal name


\usepackage{lineno} % add
\providecommand{\tightlist}{%
  \setlength{\itemsep}{0pt}\setlength{\parskip}{0pt}}

\usepackage{graphicx}
\usepackage{booktabs} % book-quality tables
%%%%%%%%%%%%%%%% end my additions to header

\usepackage[T1]{fontenc}
\usepackage{lmodern}
\usepackage{amssymb,amsmath}
\usepackage{ifxetex,ifluatex}
\usepackage{fixltx2e} % provides \textsubscript
% use upquote if available, for straight quotes in verbatim environments
\IfFileExists{upquote.sty}{\usepackage{upquote}}{}
\ifnum 0\ifxetex 1\fi\ifluatex 1\fi=0 % if pdftex
  \usepackage[utf8]{inputenc}
\else % if luatex or xelatex
  \usepackage{fontspec}
  \ifxetex
    \usepackage{xltxtra,xunicode}
  \fi
  \defaultfontfeatures{Mapping=tex-text,Scale=MatchLowercase}
  \newcommand{\euro}{€}
\fi
% use microtype if available
\IfFileExists{microtype.sty}{\usepackage{microtype}}{}
\bibliographystyle{elsarticle-harv}
\ifxetex
  \usepackage[setpagesize=false, % page size defined by xetex
              unicode=false, % unicode breaks when used with xetex
              xetex]{hyperref}
\else
  \usepackage[unicode=true]{hyperref}
\fi
\hypersetup{breaklinks=true,
            bookmarks=true,
            pdfauthor={},
            pdftitle={Comparing distance, time, and metabolic energy cost functions for walking accessibility in infrastructure-poor regions},
            colorlinks=false,
            urlcolor=blue,
            linkcolor=magenta,
            pdfborder={0 0 0}}
\urlstyle{same}  % don't use monospace font for urls

\setcounter{secnumdepth}{0}
% Pandoc toggle for numbering sections (defaults to be off)
\setcounter{secnumdepth}{0}
% Pandoc header
\usepackage{booktabs}
\usepackage{longtable}



\begin{document}
\begin{frontmatter}

  \title{Comparing distance, time, and metabolic energy cost functions for
walking accessibility in infrastructure-poor regions}
    \author[School of Geography and Earth Sciences]{Antonio Páez\corref{c1}}
   \ead{paezha@mcmaster.ca} 
   \cortext[c1]{Corresponding Author}
    \author[Department Health Research Methods Evidence and Impact]{Zoha Anjum}
   \ead{anjumz2@mcmaster.ca} 
  
    \author[Department of Civil Engineering]{Sarah E. Dickson-Anderson}
   \ead{sdickso@mcmaster.ca} 
  
    \author[Geography and Planning]{Corinne J. Schuster-Wallace}
   \ead{cschuster.wallace@usask.ca} 
  
    \author[Universidad Politécnica de Madrid]{Belén Martín Ramos}
   \ead{belen.martin@upm.es} 
  
    \author[University of Toronto Scarborough]{Christopher D. Higgins}
   \ead{cd.higgins@utoronto.ca} 
  
      \address[School of Geography and Earth Sciences]{School of Geography and Earth Sciences, McMaster University, 1280 Main
St W, Hamilton, ON, L8S 4K1 Canada}
    \address[Department Health Research Methods Evidence and Impact]{Department Health Research Methods Evidence and Impact, McMaster
University, 1280 Main St W, Hamilton, ON, L8S 4K1 Canada}
    \address[Department of Civil Engineering]{Dept. of Civil Engineering, McMaster University, 1280 Main St W,
Hamilton, ON, L8S 4K1 Canada}
    \address[Geography and Planning]{Dept. of Geography and Planning, University of Saskatchewan, Room 265
Arts, 9 Campus Drive, Saskatoon, SK, S7N 5A5 Canada}
    \address[Universidad Politécnica de Madrid]{Transport Research Centre (TRANSyT-UPM), Universidad Politécnica de
Madrid, ETSI de Caminos, Canales y Puertos, Calle Profesor Aranguren
s/n, 28040 Madrid, Spain}
    \address[University of Toronto Scarborough]{Department of Human Geography, Room HL512 1265 Military Trail Toronto,
ON, M1C 1A4 Canada}
  
  \begin{abstract}
  Accessibility is a widely used concept in transportation planning and
  research. However a majority of the literature is concerned with
  accessibility in infrastructure-rich regions where it is used to assess
  the output of infrastructure. Relatively scant attention in contrast has
  been paid to the topic of accessibility in infrastructure-poor regions.
  These are regions characterized by non-homogeneous landscapes with
  limited or no transportation infrastructure. Even studies that deal with
  infrastructure-poor regions tend to transpose the methods used
  elsewhere. This practice seems inappropriate when mobility happens by
  active rather than motorized modes since the effort required for
  movement is likely different. The objective of this paper is to compare
  distance, time, and metabolic energy cost functions in walking
  accessibility. To this end, we present a case study of accessibility to
  water in central Kenya. The results indicate that Euclidean distance,
  surface distance, and travel time correlate better between them than any
  of them does with metabolic energy. Furthermore, while shortest paths
  tend to be symmetric for distance and time criteria, under consideration
  of metabolic energy expenditure pathways change significantly depending
  on the direction of movement. This has implications for measuring
  accessibility and equity. By providing alternate mechanisms for valuing
  the cost of movement, this research suggests avenues to consider
  vulnerable populations, such as pregnant women who require greater
  nutritional intake and expend more energy per unit activity. Directions
  for further research include certain trade-offs between route choice
  variables across various applications, for example, walking and cycling
  route choice algorithms.
  \end{abstract}
  
 \end{frontmatter}




\end{document}


